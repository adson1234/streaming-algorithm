\documentclass[11pt]{article}
\usepackage{fontspec} %加這個就可以設定字體
\usepackage{xeCJK} %讓中英文字體分開設置
\usepackage{amsmath}

\setCJKmainfont{標楷體} %設定中文為系統上的字型,而英文不去更動,使用原TeX字型
\XeTeXlinebreaklocale "zh" %這兩行一定要加,中文才能自動換行
\XeTeXlinebreakskip = 0pt plus 1pt %這兩行一定要加,中文才能自動換行
\title{streaming algorithm Written Assignment $\sharp$2 }
\author{0656124 劉承順}
\date{2018/4/23} %不要日期
\begin{document}
\maketitle





\begin{enumerate}
\item 
\textbf{Algorithm}: \\
input: x \\
compute A(x) n times. save the results. \\
if the results have more than $\frac{1}{2}n$ "Yes", claim x is a "Yes-instance", output "Yes-instance". \\
otherwise,  claim x is a "No-instance", output "No-instance". \\
 \\
The time complexity is $O(log \frac{1}{\epsilon} \times$ time complexity of $A(x) )$. \\
\textbf{proof}: \\
Suppose x is a "Yes-instance" w.l.o.g. \\
Let $X_i=1$ if A(x)="Yes" at $i$th time.($X_i=0$ if A(x)="No") \\
Let $X=\Sigma X_i$. \\
Because A have 2/3 chance to be correct, Pr(A(x)="Yes")=$\frac{2}{3}$. \\
Then $E(X)=\mu=\frac{2}{3}n$. \\
By Chernoff bounds, \\
Pr[$X\leq (1-\delta)\mu$]$\leq e^{-\mu\frac{\delta^2}{2}}$. \\
Substitute $\delta$ with $\frac{1}{4}$, $\mu$ with $\frac{2}{3}n$: \\
Pr[$X\leq \frac{1}{2}n$]$\leq e^{-\frac{1}{48}n} = \frac{1}{a^n}$, for some constant $a>1$. \\
We need $\frac{1}{a^n}\leq\epsilon$, just let $n\geq log \frac{1}{\epsilon}$. \\
And we have Pr[$X\geq \frac{1}{2}n$]$\geq 1-\epsilon$, that is, the probability that this algorithm is correct is more than $1-\epsilon$. \\



\item 
\textbf{input}: stream of n elements(real number). \\
\textbf{Algorithm}: \\

for every two pass\{  //until find $k$-th smallest number\\
randomly find $n^{\frac{1}{c}}$ elements as pivots at first pass; \\

let $c_i(0\leq i \leq n^{\frac{1}{c}})$ be the number of elements which is bigger than $i$-th pivot and smaller than $i+1$-th pivot. obtain all $c_i$ at second pass. \\

suppose $\Sigma_{j=0}^{i-1} c_j < k \leq \Sigma_{j=0}^{i} c_j$, than the $k$-th smallest number must between $i$-th pivot and ${i+1}$-th pivot. so in next iteration, only consider elements between $i$-th pivot and $i+1$-th pivot. \\
\} \\

This algorithm runs O(c) pass in best case and average case. $O(n^{1-\frac{1}{c}})$ pass in worst case. 
Because every $c_i$ cost $O(log n)$ space, the total space complexity is $O(n^{\frac{1}{c}}log n)$.



\item 
(a) \\
In case $k=2$, $X_1$ and $X_2$ are pairwise independent \textit{iff} they are mutually independent.
In case $k=3$, define x$\epsilon\{1,2,3,4\}$ with same probability for each element. \\
Let $X_1=\left\{\begin{matrix}
1 & if \quad x\epsilon\{1,2\}&\\ 
0 & otherwise &
\end{matrix}\right.$ \\
$X_2=\left\{\begin{matrix}
1 & if \quad x\epsilon\{1,3\}&\\ 
0 & otherwise &
\end{matrix}\right.$ \\
$X_3=\left\{\begin{matrix}
1 & if \quad x\epsilon\{1,4\}&\\ 
0 & otherwise &
\end{matrix}\right.$ \\
we have $P(X_1=1)P(X_2=1)=\frac{1}{4}=P(X_1=1 \wedge X_2=1)$, \\
 $P(X_1=1)P(X_3=1)=\frac{1}{4}=P(X_1=1 \wedge X_3=1)$, \\
 $P(X_2=1)P(X_3=1)=\frac{1}{4}=P(X_2=1 \wedge X_3=1)$. \\
but  $P(X_1=1)P(X_2=1)P(X_3=1)=\frac{1}{8} \neq P(X_1=1 \wedge X_2=1 \wedge X_3=1)=\frac{1}{4}$. \\
So pairwise independence does not imply mutually independence. \\
In case $k \geq 4$, by induction, if case $k-1$ is proved, add new variable $X_k$. \\
Let Pr[$X_k=0$]=1, $X_k$ is independent with $X_1,...,X_{k-1}$. \\
$X_1,..X_k$ are still pairwise independent, but $X_1,X_2,X_3$ are not mutually independent. \\
So case $k$ is also proved. \\


(b) \\
cov(X,Y)=E[XY]-E[X]E[Y]. \\
If X,Y are independent,  \\
E[X]E[Y]=$[\Sigma_i p_i X_i] [\Sigma_j p_j Y_j]$ \\
=$\Sigma_{i,j} p_i p_j X_i Y_j$ \\
=$\Sigma_{i,j} Pr[X=X_i \wedge Y=Y_j] X_i Y_j$ \\
=E[XY]. \\
So, cov(X,Y)=0. \\
If $X_1$,...,$X_n$ are pairwise independent,  \\
Var(X=$\Sigma X_i$)=$\Sigma$Var($X_i$)+$\Sigma_{i\neq j}$cov($X_i$,$X_j$) \\
=$\Sigma$Var($X_i$). \\





\item 
(a) \\
Let $F$ be one of $MSF(G)$. \\
Replace $F \cap E_r$ with $F_r$.(They are both MSF($E_r$)) \\
that is, $F'=(F\setminus E_r)\cup F_r$. \\
$F'$ is $MSF(G)$, because the total weight of $F'$ holds, and $F'$ is still in G. And because both $F \cap E_r$ and $F_r$ connects all connected components of $E_r$, $F'$ is a MSF.\\
$F'$ is also $MSF(G')$, so $\mu(G) = \mu(G')$

(b) \\
Use disjoint-set forest with Union-by-rank. \\

Input: a sequence of edges $e_i = \{u_i, v_i\}$ and the edge weight $w(e_i)$. \\ 
Algorithm: \\
$ F\leftarrow \emptyset $;  \\

for each node x\{ //init disjoint-set forest\\
$p[x]\leftarrow x$,  //set parent \\
$t[x]\leftarrow 0$   //weight\\
\}; \\
for each incoming edge $e_i$ \{  \\    
	if (F $\cup$ {e} is acyclic)\{      \\     
		F $\leftarrow$ F $\cup$ {$e_i$};      \\
		Union($u_i, v_i$); \\
		If tree of $u_i$ is higher than tree of $v_i$, $ p(root(v_i))=root(u_i)$, $t[root(v_i)]=w(e_i)$, vice versa.; \\
	\}else\{           \\
		discard $e_k$ whose weight $w(e_k)$ is largest among all edges on the cycle;         \\
		//to find it, travel from $u_i$ to $root(u_i)$, and $v_i$ to $root(v_i)$,find the minimum t[x].
	\} \\
\}

output F; \\

this algorithm runs in $O(mlog n)$ time, because trees of disjoint-set forest are of O(log n) height. \\ 






\item (a)
I don't know.









\end{enumerate} 

\end{document}
