\documentclass[11pt]{article}
\usepackage{fontspec} %加這個就可以設定字體
\usepackage{xeCJK} %讓中英文字體分開設置
\usepackage{amsmath}

\setCJKmainfont{標楷體} %設定中文為系統上的字型,而英文不去更動,使用原TeX字型
\XeTeXlinebreaklocale "zh" %這兩行一定要加,中文才能自動換行
\XeTeXlinebreakskip = 0pt plus 1pt %這兩行一定要加,中文才能自動換行
\title{streaming algorithm Written Assignment $\sharp$1 }
\author{0656124 劉承順}
\date{} %不要日期
\begin{document}
\maketitle





\begin{enumerate}
\item $Pr[some\ s_i\ 
in\ S\ has\ rank\ r_A(s_i) \in [t −
\sqrt{n}
, t +
\sqrt{n}]]$\\
$ = 1-Pr[\bigcap_{s_i\in S}\ r_A(s_i) \notin [t −
\sqrt{n}
, t +
\sqrt{n}]]$\\
$ = 1-\prod_{s_i\in S}Pr[r_A(s_i) \notin [t −
\sqrt{n}
, t +
\sqrt{n}]]$ (because $s_i$'s are chosen independently)\\
$Pr[r_A(s_i) \notin [t −
\sqrt{n}
, t +
\sqrt{n}]]<\frac{n-\sqrt{n}}{n}$ for all $s_i\in S$ because $|[t −
\sqrt{n}
, t +
\sqrt{n}]|\geq\left \lfloor \sqrt{n} \right \rfloor+1$ for all t.

$\prod_{s_i\in S}Pr[r_A(s_i) \notin [t −
\sqrt{n}
, t +
\sqrt{n}]]<(1-\frac{1}{\sqrt{n}})^k$\\
because $k=\Omega(\sqrt{n})$, $\exists c>0$ and $n_0$ such that $n>n_0\Rightarrow k>c\sqrt{n}$.\\
consider function $f(x)=(1-\frac{1}{x})^x$ is increasing function when $x>1$, and 
$\lim_{x\rightarrow\infty} (1 - 1/x)^{cx} = e^{-c}$.\\
so $n>n_0\Rightarrow k>c\sqrt{n} \Rightarrow (1-\frac{1}{\sqrt{n}})^k<(1-\frac{1}{\sqrt{n}})^{c\sqrt{n}}<e^{-c}$.\\
$n<=n_0\Rightarrow (1-\frac{1}{\sqrt{n}})^k<(1-\frac{1}{\sqrt{n}})<(1-\frac{1}{\sqrt{n_0}})$.\\
then pick $\delta=min\{1-e^{-c},\frac{1}{\sqrt{n_0}}\}$, the proof is finished.\\





\item 
S is a simple graph of average degree n/3.\\
for all nodes $n_i$, $0\leq deg(n_i)\leq n-1$.\\
let $x=$ratio of nodes of degree at least n/4, $x=\frac{|\{n_i|deg(n_i)\geq\frac{n}{4}\}|}{n}$,
 $1-x=$ratio of nodes of degree less than n/4.\\
we have $\frac{n}{3}\leq(n-1)x+\frac{n}{4}(1-x)$\\
$(\frac{3}{4}n-1)x\geq\frac{n}{12}$\\
$x\geq\frac{1}{9}$ (consider $n\geq4$)\\
$|\{n_i,deg(n_i)\geq\frac{n}{4}\}|\geq\frac{1}{9}n$\\
so $|\{n_i,deg(n_i)\geq\frac{n}{4}\}| = \Omega(n)$.









\item 
Let $X=|S_n|$, $X_k=1$ if $k\in S$(with probability 1/k), $X_k=0$ if $k\notin S$.\\
we have $X=\Sigma X_k$.\\
Let$\mu=E[X]$,$\mu=\Sigma_{k\in [1,n]}\frac{1}{k}=\Theta(logn)$.\\ 
By Chernoff Bounds, we have\\
$Pr[|X-\mu|\geq\delta\mu]\leq 2e^{-\mu\delta^2/3}$ for all $0<\delta<1$\\
$Pr[|S_n|=\Theta(logn)]\geq Pr[|X-\mu|\leq0.5\mu]$\\
$\geq 1-2e^{-\mu 0.25/3}     \geq 1-\frac{1}{n^{\Omega(1)}}$


\item 
Let $G_{abc}$ be a vertex-induced subgraph of G with three different nodes $a,b,c\in V(G)$.\\
Pr[$G_{abc}$ is triangle-free]$=1-(\frac{c}{n})^3$\\
Pr[$G$ is triangle-free]=Pr[$\wedge_{G_{abc}\subseteq G} (G_{abc}$ is triangle-free)]\\
$\geq(1-(\frac{c}{n})^3)^{(^{n}_{3})}$ (because "$G_{abc}$ is triangle-free" is monotone decreasing.)\\
$\geq(1-(\frac{c}{n})^3)^{n^3}$ if $n>c$\\
because $(1-(\frac{c^3}{x}))^{x}$ is increasing function when $c>0$,\\
pick $n_0>c$ and $\delta=(1-(\frac{c}{n_0})^3)^{n_0^3}$\\
$n>n_0\Rightarrow  (1-(\frac{c}{n})^3)^{n^3}>\delta \Rightarrow$ Pr[$G$ is triangle-free]$\geq\delta$\\






\item (a)
一般解法(不限制space):答案的兩個正方形在座標圖上有兩種排法--左上加右下, 或左下加右上。因為任一個正方形的上邊必須剛好切過最高點(y值最大的點)或是下邊必須剛好切過最低點。左邊必須切過x最小點,或是右邊必須切過x最大點。兩種排法分別找最小的正方形邊長,較小的則為解答。找極值:\\
max\{x|(x,y)$\in P\}=x_{max}$ (P is the point set)\\
min\{x|(x,y)$\in P\}=x_{min}$\\
max\{y|(x,y)$\in P\}=y_{max}$\\
min\{y|(x,y)$\in P\}=y_{min}$\\
假設目前考慮兩個正方形是左下加右上的排法,令$d=$最小的正方形邊長,使正方形可以蓋住目前已處理的點。
把每一點(x,y)處理一遍:\\
$d=max(d,min\{max(x-x_{min},y-y_{min}),max(x_{max}-x,y_{max}-y)\})$\\
其中$max(x-x_{min},y-y_{min})$等於左下正方形要蓋住(x,y)所需的最小邊長,與右上正方形所需的最小邊長比較,較小者用來更新d。\\
\\
限制space的解法:以機率$p=d(1/\varepsilon)logn/n$ (for some constant d)取樣每個點,進行一般解法,得覆蓋範圍R。\\
space complexity為$O((1/\varepsilon)logn)$, time complexity為$O(n+(1/\varepsilon)logn)$.\\
$Pr[|R\cap P|\geq(1-\varepsilon)n]\\
\geq 1-Pr[|R\cap P|<(1-\varepsilon)n]$\\
$\geq 1-O(n^3)(1/e^{\varepsilon p n})$\\
$\geq 1-(1/n^{\Omega (n)})$ when d is sufficiently large.\\









\end{enumerate} 

\end{document}
